Chat applications are around our daily lives. Most of the people don't get out of their homes without checking Facebook Messenger or Whatsapp. Other popular examples, just to mention a few, are chats like Slack, Telegram, Skype, Snapchat and Viber, among many others.

We did research about the most important programming trends and the clients available for the most popular applications. We used Stack Overflow Developer Survey in 2017 ~\cite{DeveloperStackoverflow} as a source, it establishes that the most popular technologies are Android, iOS and Web. In addition, the market share of iOS and Android in 2016 is about 12.9 and 86.2 percent, respectively ~\cite{DeveloperStackoverflow}.  We also added the Web client because web development has always been attractive for us, since companies like Google, Amazon and Facebook started from there. In addition, the entry barrier in a web application is lower than other developing environments.

One of the most important reviews we research at the beginning was proper methodologies to work as a group. Specifically, the agile methodology, proposed by Alistain Cockburn and Jim Highsmith ~\cite{rury346} was our first approach. However, the lack of experience set in and we didn't know how to tackle in advance many of the software challenges laying ahead. Moreover, the proper division of each single task of our software made us struggle at some point, and we  returned to the foundations of the agile methodology. That is, using an hybrid approach on which some tests were conducted at the end and others in each single iteration. In other words, our main challenge in this point was deciding how to tackle each part of the overall system and make the integration. 

As part of the creation of this project, we ultimately decided that in order to get the most out of it, we needed to learn more about different tools. We wanted to pick a challenging technology with the robustness needed but also with vasts amounts of information on the Internet and in books to try and reduce accelerate our learning curve the most we could to mitigate our knowledge lack. 

For the clients we decided to go with Android, iOS and Web and Firebase as our back-end because we were thinking of focusing on the User Experience inside the clients. We wanted users to have a good experience using our service and convince them to use it. From the beginning, we knew that the Authentication is a strong feature in Firebase. It allows us to control who can read your data and also let us select different services such as Facebook, Gmail, GitHub, and others to register and login. One of the main things we found that Firebase has is the real time database. Thanks to this feature, we could see all messages in real time in all the clients. 

As we mentioned before, Firebase was the best choice for our intended purposes, not only because is supported by Google. Aside all the benefits such as Web Analytics, Push Notifications, File storage, and the main reason, the price (which is free for our needs) and also support scalable applications. Firebase gives the flexibility to do a lot of ``back-end" code in the client side. Even though, it leverages a lot to the client, we realised this was interesting to test new ways of developing applications, as we only knew the basics around Object Oriented Programming. It gave us also the flexibility to focus on the front-end development but with the possibility to con troll features that normally a back-end covers. Another of the main reasons why we choose Firebase is that it has a good written documentation on its website. Lastly, it takes minimum setup to begin developing, which was a key point for us as we wanted to finish the project on time while competing between studies and time. Even with this concern in mind we had some challenges finishing it but we have learn a lot in this process.
