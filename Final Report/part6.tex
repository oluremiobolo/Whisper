What worked well: We've managed to have an accelerated learning curve that allowed us to meet the requirements and do a bit of extra work. Nevertheless, we have  to say that we wasted a lot of time trying to figure out a proper mechanism to work together and integrate all the parts in a single piece. In fact, as we have mentioned previously, work integration was our main challenge. 

Working as a team involved multiple challenges. Our different backgrounds and perspectives lead us to think on creative solutions. However, to develop properly that potential, a group project structure must be followed to succeed as a team. Some minor difficulties were agenda synchronisation and conciliating different point of views.

During our execution, our plan changed a lot and we realised that planning in the initial stage of a project is essential for a good performance. In addition, we also learnt that to make a good plan that enable us to work as team, is not simple and requires experience and knowledge about software development. Its important to divide the tasks and properly allocate them. This could mean assigning each task to the person that knows most about it.

During the project we had to make many reevaluations of our plans. One good example of this is our decision of using Firebase and not create our back-end from scratch. After our first feedback, we had many evaluations to decided if it would be a good idea to continue using Firebase. This process induced more research and contrasting points of views, but finally we agreed to continue with our initial plan.

About our weakness and strengths as a team, this is a paradox in the sense that our principal strength is the diversity of professions, countries and languages, but this is also our weakness. Having different professional of multiple backgrounds, other than software engineering and computer science, gave to the group multiple perspectives and ideas. On the other hand, the deliberation process became more complex, and we lost in some situations “practical” reasoning as a group unit. In addition, these differences made the learning curve for this specific project much more challenging, but also way more rewarding at the end. 

As a group, we learnt a lot in the path and despite our knowledge disadvantage in developing, we managed to complete the main requirements for the distributed chat system. The application is far from complete in many senses. Some of the features we wanted to implement such as security and encryption can be improved and different types of messages couldn't be added, such as images, videos, Gifts and documents. In addition, the structure of our database would make it easy to add the feature of a Group Chat. 

For a production-ready chat system, this project allows us to appreciate the complexity of this kind of system, as well as all the possibilities and features that can be added. Our chat, can be considered as an ``introduction" of a chat system and the opportunity areas that can be spotted. For instance, chat rooms, survey objects, user rating, etc. This without mention features like text recognition, an AI agent that helps you with daily tasks such as weather or exchange rates, or a proper design to recognise patterns of users that allow to improve continually the application (e.g. by using A/B testing, user habits and other browsing information).
